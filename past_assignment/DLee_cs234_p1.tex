\documentclass[10pt]{article}
\usepackage[utf8]{inputenc}
\usepackage[T1]{fontenc}
\usepackage{amsmath}
\usepackage{amsfonts}
\usepackage{amssymb}
\usepackage[version=4]{mhchem}
\usepackage{stmaryrd}
\usepackage{bbold}
\usepackage{hyperref}

\title{Prep Work 1 - Functions, Strings, and Counting }

\author{CS 234}
\date{Daniel Lee}


\begin{document}
\maketitle

\section*{1 \quad Functions}

\begin{enumerate}
  \item In your own words, what is a function?

        A function is a tool connecting the input to the output.
  \item Give your own example of a function.

        The function $KtoM : \mathbb{R} \to \mathbb{R}$ defined by $KtoM(k) = k \times 0.6213711922$ maps kilometers to miles.
  \item What is a domain?

        A domain is the type of input parameter to a function.
  \item What is the domain of the function from task 2?

        $\mathbb{R}$, which is the input based on kilometers.
  \item What is a codomain?

        A codomain is the type of output of the function.
  \item What is the codomain of the function from task 2?

        $\mathbb{R}$, which is the output based on miles.
  \item Explain what the function type $\mathbb{Z}^{2} \rightarrow \mathbb{N} \times \mathbb{Q}$ means.

        The function means that it accepts a pair of integers as an input (domain) and has a pair of natural number and rational number as the output (codomain).
\end{enumerate}

\section*{2 \quad Strings}

\begin{enumerate}
  \item In your own words, what is a string?

        A string is a consectuive alignment of characters.
  \item Give your own example of a non-empty string.

        DENISONUNIVERSITYCS234
  \item What notation do we use for the length of a string?

        If there is a string s, we use two vertical bars || to denote the length of the string s as |s|.
  \item What is the length of $\epsilon$ ? (I use $\epsilon$ to represent the empty string, whereas the textbook uses $\lambda$. Either symbol is fine.)

        0
  \item In your own words, what is a substring?

        A substring is a gathering of an empty string and any strings that could constitute as a partial sequence of a string.
  \item Give an example of a substring of the string from task 2.

        UNIVERSITY
  \item In your own words, what is a prefix?

        A prefix is a gathering of strings that include an empty string and substrings that start with the initial character of a string.
  \item Give an example of a prefix of the string from task 2.

        DENISON
  \item In your own words, what is a suffix?

        A suffix is a gathering of strings that include an empty string and substrings that ends with the last character of a string.
  \item Give an example of a suffix of the string from task 2.

        CS234
  \item Is the empty string a prefix, substring, and/or suffix of the string from task 2? If so, which?

        The empty string can be a prefix, substring, and a suffix of the string from task 2 at the same time.
  \item In your own words, what is a language?


        A language is a set constituted with all existing strings over a specific alphabet as an element of that set.

  \item Give your own example of a language containing at least 3 elements.

        Over $\Sigma$ = \{x, y, z\}, the language \{s : s has length 3\} is \{xyz, xzy, yxz, yzx, zxy, zyx\}
  \item List the first 3 elements of the language from task 13 in shortlex order.

        xyz, xzy, yxz


\end{enumerate}

\section*{3 \quad Counting}

\begin{enumerate}
  \item In your own words, what is the multiplication rule?

        The multiplication rule is a rule of calculating the numbers of all possible cases that could exist.
  \item Give your own example of counting something using the multiplication rule.

        Counting all possible cases of creating a fruit dessert basket with an orange, a pineapple, and an apple.
  \item How is the factorial function mathematically defined?

        $n! = n \times (n-1) \times \ldots \times 1$
  \item What does the factorial function count?

        The number of ways of making arrangements without repeats and considering that order matters.
  \item Give your own example of counting something using factorials.

        Counting the number of ways to rearrange five newly released cars from different brands in a line for the Seoul Mobility Show.
  \item In your own words, explain how the division described in chapter A. 4 accounts for overcounting.

        The division accounts for overcounting by ruling out the cases that are able to be considered duplicates or repetition.
  \item How are binomial coefficients mathematically defined? (A binomial coefficient is just the proper term for " $n$ choose $k$ ", i.e., $\binom{n}{k}$.)

        $\binom{n}{k}=\frac{n \times(n-1) \times \ldots \times(n-k+1)}{1 \times 2 \times \ldots k}=\frac{n!}{(n-k)!k!}$
  \item What do binomial coefficients count?

        Binomial coefficients count the number of ways to choose exactly k items without repetition from among n choices.
  \item Give your own example of counting something using binomial coefficients.

        Counting the number of ways to choose the starting eleven for a soccer match from 23 players constituting the roster.
\end{enumerate}

\section*{References}
\url{https://www.unitconverters.net/length/km-to-miles.htm}

\end{document}